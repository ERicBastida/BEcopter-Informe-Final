% -*-coding: iso-latin-1  -*-
%---------------------------------------------------------------------
%
%                      resumen.tex
%
%---------------------------------------------------------------------
%
% Contiene el cap�tulo del resumen.
%
% Se crea como un cap�tulo sin numeraci�n.
%
%---------------------------------------------------------------------

\chapter{Resumen}
\cabeceraEspecial{Resumen}

\begin{FraseCelebre}
\begin{Frase}
La �nica forma de hacer un gran trabajo es amar lo que haces
\end{Frase}
\begin{Fuente}
-Steve Jobs-
\end{Fuente}
\end{FraseCelebre}

�ltimamente se ha visto que el mercado drones, es decir, los veh�culos a�reos no tripulados, del ingl�s Unmanned Aerial Vehicle (UAV) se est� expandiendo a usos cotidianos en la sociedad, a tal punto que han llegado a utilizarse en tareas tales como misiones de b�squeda y rescate, inspecci�n de l�neas el�ctricas, actividades agr�colas, seguridad vial, lucha contra incendios forestales y hasta en aspectos recreativos, como captura y transmisi�n de im�genes a�reas. Sin embargo, en la actualidad hay pocas aplicaciones que permitan a personas sin un alto grado de especializaci�n en la tem�tica implementar algoritmos. Por tal motivo y para solventar este problema, se propone implementar un software donde sea posible poner en marcha estas ideas o emprendimientos. 

De esta manera, el presente proyecto involucrar� el desarrollo de una plataforma que provea de funcionalidades a m�s alto nivel con el prop�sito de poder implementar acciones de una manera mucho m�s sencillo para el usuario. Adem�s, como caso  particular se incluir� la instrumentaci�n de un cuadric�ptero, considerando cada etapa del armado del mismo: desde el ensamblado hasta la prueba de vuelo.  Finalizando as� con una aplicaci�n sumamente personalizable y de c�digo abierto. 

\smallskip
\noindent \textbf{Palabras claves: UAVs, drone, cuadric�ptero, plataforma, c�digo abierto.} 

\endinput
% Variable local para emacs, para  que encuentre el fichero maestro de
% compilaci�n y funcionen mejor algunas teclas r�pidas de AucTeX
%%%
%%% Local Variables:
%%% mode: latex
%%% TeX-master: "../Tesis.tex"
%%% End:
