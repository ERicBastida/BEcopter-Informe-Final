% -*-coding: iso-latin-1  -*-
%---------------------------------------------------------------------
%
%                      agradecimientos.tex
%
%---------------------------------------------------------------------

\chapter{Agradecimientos}

\cabeceraEspecial{Agradecimientos}

\begin{FraseCelebre}
	\begin{Frase}
		Si caminas solo, ir�s m�s r�pido; si caminas acompa�ado, llegar�s m�s lejos.
	\end{Frase}
	\begin{Fuente}
		Proverbio chino
	\end{Fuente}
\end{FraseCelebre}

Se esta terminando un gran viaje, y como en todo gran aventura vas aprendiendo much�simas cosas en el transcurso del camino, haci�ndote mejor persona. En este viaje he conocido compa�eros de ruta que han aportado en mi, mucho de ellos. Es por eso que estoy realmente agradecido a todos los profesores que he tenido en cada materia que tiene esta hermosa carrera que he elegido; por sus conocimientos, por su paciencia y por sobre todo sus ganas de ense�ar y de que nosotros aprendamos, ya que no es una tarea f�cil. Adem�s de docentes, quiero agradecer a mis directores por estar siempre disponibles cuando necesite ayuda para realizar este proyecto y en especial a la Dra. Marina Murillo que m�s all� de no aparecer en el documento como directora, fue incondicional brind�ndome su ayuda en gran parte del desarrollo del proyecto. 

No fue un camino f�cil, hubieron momentos de incertidumbre, miedos, nervios y desesperaci�n. Por suerte, existieron personas que compartieron tambi�n gran parte del camino, y que actualmente, se han convertido en amigos de futuros viajes, por lo que les agradezco todo lo compartido hasta el momento. Y por �ltimo quiero agradecer profundamente a mi querida familia que siempre estuvo apoy�ndome en las ganas de seguir este viaje que esta pronto a culminarse y en especial, quiero dedicar todo el esfuerzo realizado, a mi querida madre Liliana, que gracias a ella soy la persona que soy ahora, que siempre me inculc� en mi  \textit{que de la necesidad sale el \textit{ingenio}, de busc�rsela para solucionar cualquier problema que se presente} y es por tal raz�n que he elegido la ingenier�a, as� que todo esto es gracias a vos madre.

\endinput

% Variable local para emacs, para  que encuentre el fichero maestro de
% compilaci�n y funcionen mejor algunas teclas r�pidas de AucTeX
%%%
%%% Local Variables:
%%% mode: latex
%%% TeX-master: "../Tesis.tex"
%%% End:
