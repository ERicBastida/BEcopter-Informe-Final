% -*-coding: iso-latin-1  -*-
%---------------------------------------------------------------------
%
%                      agradecimientos.tex
%
%---------------------------------------------------------------------
%
% agradecimientos.tex
% Copyright 2009 Marco Antonio Gomez-Martin, Pedro Pablo Gomez-Martin
%
% This file belongs to the TeXiS manual, a LaTeX template for writting
% Thesis and other documents. The complete last TeXiS package can
% be obtained from http://gaia.fdi.ucm.es/projects/texis/
%
% Although the TeXiS template itself is distributed under the 
% conditions of the LaTeX Project Public License
% (http://www.latex-project.org/lppl.txt), the manual content
% uses the CC-BY-SA license that stays that you are free:
%
%    - to share & to copy, distribute and transmit the work
%    - to remix and to adapt the work
%
% under the following conditions:
%
%    - Attribution: you must attribute the work in the manner
%      specified by the author or licensor (but not in any way that
%      suggests that they endorse you or your use of the work).
%    - Share Alike: if you alter, transform, or build upon this
%      work, you may distribute the resulting work only under the
%      same, similar or a compatible license.
%
% The complete license is available in
% http://creativecommons.org/licenses/by-sa/3.0/legalcode
%
%---------------------------------------------------------------------
%
% Contiene la p�gina de agradecimientos.
%
% Se crea como un cap�tulo sin numeraci�n.
%
%---------------------------------------------------------------------

\chapter{Agradecimientos}

\cabeceraEspecial{Agradecimientos}

\begin{FraseCelebre}
	\begin{Frase}
		Si caminas solo, ir�s m�s r�pido; si caminas acompa�ado, llegar�s m�s lejos.
	\end{Frase}
	\begin{Fuente}
		Proverbio chino
	\end{Fuente}
\end{FraseCelebre}

%Se esta terminando un gran viaje, y como en todo gran viaje vas aprendiendo much�simas cosas en el transcurso del camino haciendote mejor persona. Y en el transcurso de este viaje he conocido compa�eros de ruta que han aportado en mi, mucho de ellos. Es por eso que estoy realmente agradecido a todos los profesores que he tenido en cada materia que tiene esta hermosa carrera que he elegido; por sus conocimientos, por su paciencia y por sobre todo sus ganas de ense�ar y de que nosotros aprendamos, ya que no es una tarea f�cil. Adem�s de docentes, quiero agradecer a mis directores por estar siempre disponibles cuando necesite ayuda para realizar este proyecto y en especial a la Dra. Marina Murillo que m�s all� de no aparecer en el documento como directora, fue incondicional brindandome su ayuda en gran parte del desarrollo del proyecto. 

% No fue un camino f�cil, hubieron momentos de incertidumbre, miedos, nervios y desesperaci�n. Por suerte, existieron personas que compartieron tamb�en gran parte del camino, y que actualmente, se han convertido en amigos de la vida, por lo que les agradezco todo lo compartido hasta el momento. Y por �ltimo quiero agredecer profundamente a mi querida familia que siempre estuvo apoyandome en las ganas de seguir este viaje que esta pronto a culminarse y en especial, quiero dedicar todo el esfuerzo realizado hasta el ahora, a mi querida madre Liliana, que gracias a ella soy la persona que soy ahora, que siempre me inculc� en mi  "que de la necesidad sale el \textit{ingenio}, de rebuscarsela para solucionar cualquier problema que se presente" y es por tal raz�n que he elegido la ingenieria, as� que todo esto es gracias a vos madre.

\endinput

% Variable local para emacs, para  que encuentre el fichero maestro de
% compilaci�n y funcionen mejor algunas teclas r�pidas de AucTeX
%%%
%%% Local Variables:
%%% mode: latex
%%% TeX-master: "../Tesis.tex"
%%% End:
